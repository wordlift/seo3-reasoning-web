% ---------------------------------------------------------------------------
% Structured Linked Data for Agentic RAG:
% Boosting Retrieval with Knowledge Graphs
%
% Target: ISWC 2026 Research Track — LNCS format
% ---------------------------------------------------------------------------
\documentclass[runningheads]{llncs}

\usepackage[utf8]{inputenc}
\usepackage[T1]{fontenc}
\usepackage{graphicx}
\usepackage{amsmath}
\usepackage{booktabs}
\usepackage{hyperref}
\usepackage{xcolor}
\usepackage{listings}
\usepackage{subcaption}
\usepackage{float}

\begin{document}

\title{Structured Linked Data as a Memory Layer\\
for Agent-Orchestrated Retrieval}

\titlerunning{Structured Linked Data for Agent-Orchestrated Retrieval}

\author{Andrea Volpini}

\authorrunning{Volpini}

\institute{WordLift, Rome, Italy\\
\email{andrea@wordlift.io}}

\maketitle

% ===========================================================================
\begin{abstract}
Retrieval-Augmented Generation (RAG) systems typically treat documents as
flat text, ignoring the structured metadata and linked relationships that
knowledge graphs provide. In this paper, we investigate whether structured
linked data---specifically Schema.org markup and dereferenceable entity
pages served by a Linked Data Platform---can improve retrieval accuracy
and answer quality in both standard and agentic RAG systems.

We conduct a controlled experiment across four domains (editorial, legal,
travel, e-commerce) using Vertex AI Vector Search 2.0 for retrieval and
the Google Agent Development Kit (ADK) for agentic reasoning. Our
experimental design tests seven conditions: three document representations
(plain HTML, HTML with JSON-LD, and an enhanced agentic-optimized entity
page) crossed with two retrieval modes (standard RAG and agentic RAG
with multi-hop link traversal), plus an Enhanced+ condition that adds
rich navigational affordances and entity interlinking.

Our results reveal that while JSON-LD markup alone provides only modest
improvements ($\Delta = +0.17$, $p_{\text{adj}} = 0.024$), our enhanced entity page
format---incorporating \texttt{llms.txt}-style agent instructions,
breadcrumbs, and neural search capabilities---achieves substantial
gains: +29.6\% accuracy improvement for standard RAG ($p < 10^{-21}$,
$d = 0.60$) and +29.8\% for the full agentic pipeline ($p < 10^{-21}$,
$d = 0.61$). The Enhanced+ condition achieves the highest absolute
scores (accuracy: 4.85/5, completeness: 4.55/5), representing a
+34.0\% improvement over the baseline ($p < 10^{-24}$, $d = 0.65$).
We release our dataset, evaluation framework,
and enhanced entity page templates to support reproducibility.

\keywords{Retrieval-Augmented Generation \and Knowledge Graphs \and
Linked Data \and Structured Data \and Schema.org \and
Agentic AI \and Vector Search}
\end{abstract}

% ===========================================================================
\section{Introduction}
\label{sec:intro}

The rise of Generative AI has fundamentally changed how users access
information online. Search engines increasingly augment traditional
results with AI-generated summaries---a paradigm exemplified by
Google's AI Mode, which retrieves, reasons over, and synthesizes
information from multiple web sources. Understanding and optimizing
for this new retrieval paradigm is critical for content creators,
marketers, and organizations that depend on search visibility.

Retrieval-Augmented Generation (RAG) has emerged as the dominant
architecture for grounding large language model (LLM) outputs in
factual, up-to-date information~\cite{lewis2020rag}. However, most
RAG implementations treat documents as unstructured text, discarding
the rich structured metadata that many websites already provide via
Schema.org markup and knowledge graph representations.

In this paper, we ask: \textbf{Can structured linked data improve
RAG performance, and does agentic link traversal unlock further gains?}

Our work is motivated by three observations:
\begin{enumerate}
    \item Websites increasingly embed Schema.org JSON-LD structured data,
          yet RAG systems rarely exploit this metadata.
    \item Linked Data Platforms serve entity pages that support content
          negotiation, enabling programmatic traversal of knowledge graphs.
    \item Agentic AI systems (those capable of planning, tool use, and
          multi-step reasoning) can follow links and aggregate information
          across entity boundaries---mimicking the behavior of AI-powered
          search engines.
\end{enumerate}

We make the following contributions:
\begin{itemize}
    \item A controlled experimental framework comparing seven conditions
          (3 document formats $\times$ 2 retrieval modes + an Enhanced+
          variant) across four industry verticals, with 2,443 individual
          query evaluations.
    \item An \emph{enhanced entity page} format designed to maximize
          both human readability and agentic discoverability, incorporating
          \texttt{llms.txt}-style instructions and neural search capabilities,
          and an Enhanced+ variant with richer navigational affordances.
    \item Empirical evidence showing that enhanced entity pages yield
          the strongest improvements: +29.6\% accuracy in standard RAG
          ($d = 0.60$) and +34.0\% in the full Enhanced+ agentic pipeline
          ($d = 0.65$), while JSON-LD markup alone provides only
          marginal improvements.
    \item A reusable dataset and evaluation harness released for
          reproducibility.
\end{itemize}

% ===========================================================================
\section{Related Work}
\label{sec:related}

\subsection{Generative Engine Optimization}

Aggarwal et al.~\cite{aggarwal2023geo} introduced Generative Engine
Optimization (GEO), demonstrating that content optimization strategies
such as adding citations, statistics, and authoritative language can
boost visibility in generative search engines by up to 40\%.
Our work extends GEO from \emph{visibility optimization} to
\emph{retrieval accuracy}, focusing on structured data as the
optimization lever.

\subsection{Retrieval-Augmented Generation}

RAG was formalized by Lewis et al.~\cite{lewis2020rag}, who combined
a pre-trained sequence-to-sequence model with a dense retriever to
ground generation in retrieved passages. Subsequent work explored
pre-training with retrieval objectives~\cite{guu2020realm} and
scaling retrieval corpora to trillions of tokens~\cite{borgeaud2022retro}.
More recently, Self-RAG~\cite{asai2024selfrag} introduced self-reflection
mechanisms for adaptive retrieval, enabling models to decide when and
what to retrieve. Trivedi et al.~\cite{trivedi2023interleaving}
demonstrated that interleaving retrieval with chain-of-thought reasoning
significantly improves multi-step question answering.

Despite these advances, existing RAG systems predominantly operate on
unstructured text. Our work bridges this gap by demonstrating that
structured metadata---specifically Schema.org JSON-LD---provides
a complementary signal that improves retrieval quality.

\subsection{Knowledge Graphs and Structured Data on the Web}

The vision of a machine-readable web was articulated by
Berners-Lee et al.~\cite{berners2001semantic} and operationalized
through Linked Data principles~\cite{bizer2009linked}. Schema.org,
launched in 2011 as a collaboration among major search engines,
provides a shared vocabulary for structured data on the
web~\cite{guha2016schemaorg,schemaorg}. Today, over 40\% of
web pages include Schema.org markup~\cite{guha2016schemaorg}.

Knowledge graphs have become central to both academic research and
industry applications~\cite{hogan2021knowledge,noy2019industry}.
Early efforts to bring structured data to content management systems
include WordLift~\cite{volpini2015wordlift}, which introduced semantic
annotation and entity-based navigation for WordPress sites, and
MICO~\cite{aichroth2015mico}, which developed linked-data pipelines
for multimedia content enrichment.
Recent surveys examine the unification of LLMs and knowledge
graphs~\cite{pan2024graphrag}, while Graph RAG approaches
explicitly leverage graph structure during
retrieval~\cite{peng2024graphretrieval}. Our work differs from
Graph RAG in that we do not require a purpose-built graph database;
instead, we leverage \emph{existing} structured data already published
on the web via Schema.org and Linked Data Platforms.

\subsection{Agentic AI and Tool-Augmented LLMs}

Agentic AI systems extend LLMs with the ability to plan, use tools,
and reason over multiple steps. Yao et al.~\cite{yao2023react}
introduced ReAct, interleaving reasoning traces with action steps.
Schick et al.~\cite{schick2023toolformer} demonstrated that LLMs
can learn to use external tools autonomously. The Google Agent
Development Kit (ADK)~\cite{googleadk2025} provides a production
framework for building multi-tool agents.

Multi-hop question answering~\cite{mavi2024multihop}---where
answering requires combining information from multiple
sources---is a natural application for agentic systems. The
Model Context Protocol (MCP)~\cite{mcp2024} provides a standardized
interface for LLM--tool integration. Our
agentic RAG configuration enables link traversal across entity
boundaries, effectively mimicking the behavior of AI-powered
search systems that follow links to aggregate information.

% ===========================================================================
\section{Methodology}
\label{sec:method}

\subsection{System Architecture and AI Mode Parallel}

Our experimental system mirrors the emerging architecture of AI-powered
search engines such as Google's AI Mode, which retrieves web pages,
reasons over their structured content, and synthesizes multi-source
answers.  Our pipeline reproduces this pattern using production Google
Cloud components:

\begin{itemize}
    \item \textbf{Vertex AI Vector Search 2.0}~\cite{vertexai2025} serves
          as the retrieval backbone.  Unlike traditional vector databases,
          Vector Search 2.0 is designed as a self-tuning, fully managed,
          AI-native search engine.  It combines dense semantic search
          (via \texttt{text-embedding-005} embeddings) with sparse keyword
          matching in a single hybrid query, automatically tuning retrieval
          parameters.  This mirrors how AI Mode identifies candidate web
          pages from a large corpus.
    \item \textbf{Google Agent Development Kit (ADK)}~\cite{googleadk2025} powers
          the agentic reasoning layer, providing a ReAct-style
          loop~\cite{yao2023react} with tool-use capabilities.  Like AI
          Mode's multi-step reasoning, our agent can plan a sequence of
          actions---search, follow links, search the knowledge graph---before
          synthesizing a final answer.
    \item \textbf{WordLift Knowledge Graph} acts as the structured data
          layer, providing Schema.org-typed entities with dereferenceable
          URIs that support content negotiation
          (\texttt{application/ld+json}, \texttt{text/turtle},
          \texttt{text/html}).  This is analogous to how AI Mode leverages
          structured data already present in web pages to enhance
          understanding.
\end{itemize}

The key insight is that \textbf{structured linked data functions as an
external memory layer} for the agent.  Rather than relying solely on the
vector store's flat text chunks, the agent can follow typed relationships
(\texttt{schema:about}, \texttt{schema:author},
\texttt{schema:relatedLink}) to discover contextually relevant
information that would be invisible to embedding-based retrieval alone.

\subsection{Research Design}

We design a $3 \times 2$ factorial experiment crossing three document
representations with two retrieval modes, yielding six core experimental
conditions, plus an Enhanced+ variant (Table~\ref{tab:conditions}).

\begin{table}[t]
\centering
\caption{Experimental conditions.}
\label{tab:conditions}
\begin{tabular}{@{}clll@{}}
\toprule
\textbf{ID} & \textbf{Document Format} & \textbf{Retrieval Mode} & \textbf{Hypotheses} \\
\midrule
C1 & Plain HTML       & Standard RAG & H1 baseline \\
C2 & HTML + JSON-LD   & Standard RAG & H1 treatment \\
C3 & Enhanced entity  & Standard RAG & H3 baseline \\
C4 & Plain HTML       & Agentic RAG  & H2 baseline \\
C5 & HTML + JSON-LD   & Agentic RAG  & H2 treatment \\
C6 & Enhanced entity  & Agentic RAG  & H2+H3 treatment \\
\midrule
C6+ & Enhanced+ entity & Agentic RAG  & H4 treatment \\
\bottomrule
\end{tabular}
\end{table}

Our four hypotheses are:
\begin{itemize}
    \item \textbf{H1}: Adding Schema.org JSON-LD to HTML documents
          improves RAG accuracy and completeness (C2 vs.\ C1).
    \item \textbf{H2}: Agentic RAG with link traversal outperforms
          standard RAG on the same document format (C5 vs.\ C2).
    \item \textbf{H3}: Enhanced entity pages, designed for agentic
          discoverability, yield the highest overall performance
          (C6 vs.\ all other conditions).
    \item \textbf{H4}: Enhanced+ entity pages---with richer navigational
          affordances and entity interlinking---further improve
          performance over the base enhanced format
          (C6+ vs.\ C6).
\end{itemize}

\subsection{Document Representations}

\paragraph{Plain HTML (Baseline).}
Raw webpage content with all \texttt{<script type="application/ld+json">}
blocks removed. This represents the content as a standard RAG system
would encounter it---purely unstructured HTML.

\paragraph{HTML + JSON-LD.}
The original webpage with embedded Schema.org JSON-LD served by the
Linked Data Platform's data API. This representation includes typed
entities, properties (e.g., name, description, offers, geo-coordinates),
and inter-entity links expressed as dereferenceable URIs.

\paragraph{Enhanced Entity Page.}
A novel format designed to maximize agentic discoverability:
\begin{itemize}
    \item Natural language summary generated from structured data
    \item Embedded JSON-LD block with full Schema.org typing
    \item Visible linked entity navigation with dereferenceable URIs
    \item \texttt{llms.txt}-style agent instructions~\cite{llmstxt2024}
          providing explicit guidance for LLM agents
    \item Neural search SKILL reference for cross-entity discovery
    \item Schema.org type breadcrumbs for hierarchical context
\end{itemize}

The enhanced entity page format is designed to bridge the gap between
human-readable webpages and machine-readable structured data by making
entity relationships, navigation paths, and available tools explicitly
visible to both humans and AI agents.

\begin{figure}[H]
\centering
\includegraphics[width=\textwidth]{fig_entity_comparison.png}
\caption{Before and after: plain HTML (left) vs.\ enhanced entity page (right)
for a sample entity. The enhanced format adds structured breadcrumbs, related
entity links with dereferenceable URIs, agent instructions in
\texttt{llms.txt} style, and an embedded JSON-LD block---yielding a +29.6\%
accuracy improvement in standard RAG and +34.0\% in the full
Enhanced+ agentic pipeline.}
\label{fig:entity_comparison}
\end{figure}

\begin{figure}[H]
\centering
\includegraphics[width=\textwidth]{fig_architecture.png}
\caption{System architecture. User queries are processed by a Google ADK agent
that orchestrates three tools: vector search over Vertex AI, entity link
traversal, and neural search via MCP. Documents are indexed in three
formats (C1--C6) and the agent generates grounded answers using a
ReAct-style reasoning loop.}
\label{fig:architecture}
\end{figure}

\subsection{Retrieval Modes}

\paragraph{Standard RAG.}
Documents are indexed in Vertex AI Vector Search 2.0~\cite{vertexai2025}
using the \texttt{gemini-embedding-001} model with hybrid search
(semantic + keyword). For each query, the top-$K$ ($K=10$) documents are
retrieved and passed to Gemini 2.5 Flash~\cite{gemini2024} for answer
generation in a single inference call.

\paragraph{Agentic RAG.}
Built on the Google Agent Development Kit (ADK)~\cite{googleadk2025},
the agent operates in a ReAct-style loop~\cite{yao2023react} with access
to three tools:
\begin{enumerate}
    \item \texttt{search\_documents}: Performs vector search over the
          Vertex AI collection for semantic retrieval.
    \item \texttt{follow\_entity\_link}: Dereferences a linked entity
          URI via HTTP content negotiation (requesting JSON-LD), enabling
          multi-hop traversal of the knowledge graph.
    \item \texttt{search\_knowledge\_graph}: Performs neural search
          across the knowledge graph using a domain-specific API endpoint.
\end{enumerate}

The agent can follow links up to 2~hops deep and makes an average of
2.0 tool calls per query. This architecture effectively replicates the
behavior of production AI-powered search systems such as Google AI Mode,
making our findings directly relevant to practitioners optimizing content
for AI-driven search discovery.

\subsection{Dataset}
\label{sec:dataset}

Our dataset spans four industry verticals, chosen to test generalizability
across diverse content types and knowledge graph structures:

\begin{itemize}
    \item \textbf{WordLift Blog} (editorial): 7 entities, 21 queries.
          Blog articles about SEO, knowledge graphs, and AI content.
    \item \textbf{Express Legal Funding} (legal): 23 entities, 69 queries.
          Legal concepts including pre-settlement funding, personal injury,
          structured settlements, and regulatory topics.
    \item \textbf{SalzburgerLand} (travel): 26 entities, 78 queries.
          Restaurants, alpine huts, and tourist establishments in the
          Salzburg region of Austria.
    \item \textbf{BlackBriar} (e-commerce): 46 entities, 137 queries.
          Outdoor gear products with detailed offers, pricing, and
          product specifications.
\end{itemize}

In total, the dataset comprises \textbf{102 entities} and \textbf{349
test queries}. Entities were collected from four Linked Data Platforms
using GraphQL-based entity search, yielding structured data in JSON-LD
format with Schema.org typing. Each entity was transformed into three
document variants (plain HTML, HTML+JSON-LD, enhanced entity page) plus
an Enhanced+ variant, and ingested into separate Vertex AI Vector Search
2.0 collections.

Test queries were generated using a hybrid approach: template-based
generation for factual, relational, and comparative query types, augmented
by LLM-generated queries (Gemini 2.0 Flash) that explore entity
relationships and cross-entity reasoning. All seven conditions are
evaluated on the identical set of 349 queries, ensuring fair
apples-to-apples comparison.

\subsection{Evaluation Metrics}

All responses are evaluated by an independent LLM judge (Gemini 3.0 Flash) using
three metrics:

\begin{itemize}
    \item \textbf{Accuracy} (1--5): Factual correctness of the generated
          answer, assessed against the retrieved context and query intent.
    \item \textbf{Completeness} (1--5): Degree to which the answer covers
          all aspects of the query, including related entities and
          contextual details.
    \item \textbf{Grounding} (binary 0/1): Whether the answer is
          faithfully grounded in the retrieved documents, without
          hallucinated content. Measured for standard RAG only (C1--C3),
          as agentic RAG retrieval boundaries are less well-defined.
\end{itemize}

For agentic conditions (C4--C6, C6+), we additionally track:
\begin{itemize}
    \item \textbf{Links followed}: Number of entity links dereferenced
    \item \textbf{Links available}: Number of discoverable links in
          retrieved documents
    \item \textbf{Max hop depth}: Maximum traversal depth reached
    \item \textbf{Tool calls}: Total number of tool invocations
\end{itemize}

Statistical significance is assessed using paired $t$-tests with
Bonferroni correction across 12 comparisons ($\alpha = 0.05$),
with effect sizes reported as Cohen's $d$.

% ===========================================================================
\section{Results}
\label{sec:results}

We executed the full experiment: 349 queries $\times$ 7 conditions =
2,443 individual evaluations, yielding 2,439 valid results after
excluding error cases (1 in C4, 3 in C5). Table~\ref{tab:main_results}
presents the main results and Figure~\ref{fig:condition_bars} visualizes
the per-condition scores.

\begin{table}[t]
\centering
\caption{Main results across experimental conditions (mean $\pm$ std).}
\label{tab:main_results}
\begin{tabular}{@{}clcc@{}}
\toprule
\textbf{ID} & \textbf{Condition} & \textbf{Accuracy} & \textbf{Completeness} \\
\midrule
C1 & Plain HTML, Std. & 3.62 $\pm$ 1.82 & 3.01 $\pm$ 1.94 \\
C2 & HTML+JSON-LD, Std. & 3.89 $\pm$ 1.70 & 3.33 $\pm$ 1.85 \\
C3 & Enhanced, Std. & 4.69 $\pm$ 0.95 & 4.45 $\pm$ 1.25 \\
\midrule
C4 & Plain HTML, Agent. & 4.36 $\pm$ 1.33 & 3.98 $\pm$ 1.60 \\
C5 & HTML+JSON-LD, Agent. & 4.40 $\pm$ 1.21 & 4.00 $\pm$ 1.54 \\
C6 & Enhanced, Agent. & 4.70 $\pm$ 0.82 & 4.38 $\pm$ 1.20 \\
\midrule
C6+ & Enhanced+, Agent. & \textbf{4.85} $\pm$ \textbf{0.50} & \textbf{4.55} $\pm$ \textbf{1.06} \\
\bottomrule
\end{tabular}
\end{table}

\begin{figure}[H]
\centering
\includegraphics[width=\textwidth]{figures/condition_bars.pdf}
\caption{Mean accuracy and completeness scores by experimental condition.
Enhanced entity pages (C3, C6, C6+) dramatically outperform plain HTML and
JSON-LD conditions. C6+ achieves the highest scores. Error bars show
95\% confidence intervals.}
\label{fig:condition_bars}
\end{figure}

\subsection{Qualitative Comparison of Generated Answers}

To illustrate how structured data and agentic retrieval affect response
quality, Table~\ref{tab:qualitative} presents two representative queries
with a summary of how different conditions respond.  These examples show
how the system progressively improves from vague or generic answers (C1)
to fully accurate, entity-grounded responses (C6).

\begin{table}[H]
\centering
\small
\caption{Qualitative comparison of generated answers across conditions
for two representative queries.  Accuracy scores are assigned by the
LLM judge (1--5 scale).}
\label{tab:qualitative}
\renewcommand{\arraystretch}{1.25}
\setlength{\tabcolsep}{4pt}
\begin{tabular}{@{}p{1.4cm}p{8.2cm}r@{}}
\hline
\textbf{Cond.} & \textbf{Answer Summary} & \textbf{Acc.} \\
\hline
\multicolumn{3}{@{}l}{\textit{Factual --- ``What is Restaurant im Hotel
Zauchenseehof? Describe its key features.''}} \\
\hline
C1 & Generic description without specifics. No cuisine type,
location, or opening hours mentioned. & 1 \\
C2 & Identifies it as a FoodEstablishment with some
Schema.org properties. & 3 \\
C4 & Agent searches but finds limited structured data to
traverse. & 2 \\
\textbf{C6} & \textbf{Agent follows links to related entities
(hotel, region); retrieves cuisine, address, coordinates, and related
attractions.} & \textbf{5} \\
C6+ & Enhanced+ agent instructions and richer linked-entity
navigation surface the same information with lower variance; answer
matches C6 quality. & 5 \\
\hline
\multicolumn{3}{@{}l}{\textit{Relational --- ``What entities are related
to Google Lens?''}} \\
\hline
C1 & Vague mention of ``image recognition'' without specific
entity relationships. & 1 \\
C3 & Lists entities from the enhanced page's Related Entities
section. & 4 \\
C5 & Agent follows some links but limited by JSON-LD's
implicit relationships. & 3 \\
\textbf{C6} & \textbf{Agent uses \texttt{search\_knowledge\_graph} +
\texttt{follow\_entity\_link} to discover and traverse related entities
across the graph.} & \textbf{5} \\
C6+ & Enhanced+ provides wider linked-entity surface and
\texttt{llms.txt} agent guidance; discovers additional related entities
beyond C6. & 5 \\
\hline
\end{tabular}
\end{table}

These examples illustrate the key mechanism: enhanced entity pages
provide \emph{navigational affordances}---visible links, agent
instructions, and neural search capability---that enable the agentic
system to discover and integrate information that flat-text retrieval
misses entirely.  Figure~\ref{fig:qualitative} visualizes this
progressive improvement.

\begin{figure}[H]
\centering
\includegraphics[width=\textwidth]{fig_qualitative_comparison.png}
\caption{Answer quality progression across conditions for the same
factual query.  C1 (plain HTML) produces a generic answer (1/5),
while C6 and C6+ (enhanced entity pages + agentic RAG) follow links to
related entities and retrieve comprehensive structured data (5/5).
C6+ achieves the same peak score with lower variance.}
\label{fig:qualitative}
\end{figure}

\subsection{H1: JSON-LD Alone Does Not Significantly Help}

\label{sec:h1}

Adding JSON-LD structured data to HTML documents yields a small but
statistically significant improvement in accuracy, though the effect
size is negligible:

\begin{itemize}
    \item \textbf{Accuracy}: C2 (3.89) vs.\ C1 (3.62), $\Delta = +0.17$,
          $t = -3.12$, $p_\text{adj} = 0.024$, $d = 0.18$ (small).
    \item \textbf{Completeness}: C2 (3.33) vs.\ C1 (3.01),
          $\Delta = +0.18$, not significant after Bonferroni correction
          ($p_\text{adj} = 0.055$).
\end{itemize}

While statistically significant for accuracy, the small effect size
($d = 0.18$) suggests that appending JSON-LD metadata provides only
marginal additional signal for RAG systems when the underlying text
content already conveys similar information. The structured data must
be presented in a way that explicitly highlights entity relationships
and navigational affordances---as our enhanced entity page format does.

However, enhanced entity pages (C3) yield dramatic improvements:
accuracy 4.69 vs.\ 3.62 ($\Delta = +1.04$, $p < 10^{-21}$, $d = 0.60$),
representing a \textbf{+29.6\% improvement} with a medium effect size.

\subsection{H2: Agentic RAG Amplifies Structured Data Gains}
\label{sec:h2}

Comparing agentic RAG (C5) with standard RAG (C2) on the same
HTML+JSON-LD documents:

\begin{itemize}
    \item \textbf{Accuracy}: C5 (4.40) vs.\ C2 (3.89), $\Delta = +0.50$,
          $t = -5.22$, $p_\text{adj} = 4.0 \times 10^{-6}$, $d = 0.30$.
    \item \textbf{Completeness}: C5 (4.00) vs.\ C2 (3.33),
          $\Delta = +0.74$, $t = -6.75$,
          $p_\text{adj} = 9.1 \times 10^{-10}$, $d = 0.38$.
\end{itemize}

Agentic RAG significantly improves both accuracy (+13.1\%) and
completeness (+20.1\%). The agent's multi-step reasoning and
tool use improve both the \emph{precision} and \emph{coverage} of
answers, confirming that agentic link traversal provides meaningful
additional value over single-pass retrieval.

\subsection{H3: Enhanced Entity Pages Yield the Strongest Gains}
\label{sec:h3}

The enhanced entity page format produces the largest improvements
across both retrieval modes:

\paragraph{Standard RAG (C3 vs.\ C1):}
\begin{itemize}
    \item Accuracy: 4.69 vs.\ 3.62, $\Delta = +1.04$,
          $p = 2.7 \times 10^{-21}$, $d = 0.60$ (medium).
    \item Completeness: 4.45 vs.\ 3.01, $\Delta = +1.42$,
          $p = 5.0 \times 10^{-30}$, $d = 0.74$ (medium--large).
\end{itemize}

\paragraph{Agentic RAG (C6 vs.\ C5):}
\begin{itemize}
    \item Accuracy: 4.70 vs.\ 4.40, $\Delta = +0.34$,
          $p_\text{adj} = 7.7 \times 10^{-6}$, $d = 0.29$ (small).
    \item Completeness: 4.38 vs.\ 4.00, $\Delta = +0.40$,
          $p_\text{adj} = 2.0 \times 10^{-5}$, $d = 0.28$ (small).
\end{itemize}

\paragraph{Full pipeline (C6 vs.\ C1):}
\begin{itemize}
    \item Accuracy: 4.70 vs.\ 3.62, $\Delta = +1.04$,
          $p = 1.0 \times 10^{-21}$, \textbf{$d = 0.61$} (medium).
    \item Completeness: 4.38 vs.\ 3.01, $\Delta = +1.34$,
          $p = 5.1 \times 10^{-31}$, \textbf{$d = 0.75$} (medium--large).
\end{itemize}

\paragraph{Enhanced+ pipeline (C6+ vs.\ C1):}
\begin{itemize}
    \item Accuracy: 4.85 vs.\ 3.62, $\Delta = +1.10$,
          $p = 2.3 \times 10^{-24}$, \textbf{$d = 0.65$} (medium).
    \item Completeness: 4.55 vs.\ 3.01, $\Delta = +1.41$,
          $p = 2.8 \times 10^{-32}$, \textbf{$d = 0.77$} (medium--large).
\end{itemize}

C6+ achieves the highest scores across all conditions (accuracy:
4.85/5, completeness: 4.55/5), representing a \textbf{+34.0\%
accuracy improvement} and \textbf{+51.2\% completeness improvement}
over the baseline. Notably, while C6+ outperforms C6 in absolute terms,
the C6 $\rightarrow$ C6+ difference is not statistically significant
($\Delta = +0.06$, $p_\text{adj} = 1.0$, $d = 0.08$), suggesting that
the base enhanced entity page format captures most of the benefit.
The medium--large effect sizes ($d \approx 0.65$--$0.77$) confirm that
enhanced entity pages consistently outperform all other document formats.

\begin{figure}[H]
\centering
\includegraphics[width=0.85\textwidth]{figures/improvement_waterfall.pdf}
\caption{Accuracy improvement waterfall showing the cumulative effect of
each optimization layer: JSON-LD markup, agentic retrieval, and enhanced
entity pages. The largest gains come from enhanced page presentation,
not from adding structured data alone.}
\label{fig:waterfall}
\end{figure}

\begin{table}[t]
\centering
\caption{Paired $t$-tests with Bonferroni correction ($\alpha = 0.05$, $n_{\text{tests}} = 12$).}
\label{tab:stat_tests}
\begin{tabular}{@{}llrrrrl@{}}
\toprule
\textbf{Hyp.} & \textbf{Metric} & \textbf{$\Delta$} & \textbf{$t$} & \textbf{$p_{\text{adj}}$} & \textbf{$d$} & \textbf{Sig.} \\
\midrule
\multicolumn{7}{l}{\textit{Structured data (standard RAG) (C1 vs C2)}} \\
  & Accuracy & +0.17 & -3.12 & 2.4e-02 & 0.18 & * \\
  & Completeness & +0.18 & -2.86 & 5.5e-02 & 0.16 & n.s. \\
\midrule
\multicolumn{7}{l}{\textit{Enhanced pages (standard RAG) (C1 vs C3)}} \\
  & Accuracy & +1.04 & -10.53 & 2.7e-21 & 0.60 & *** \\
  & Completeness & +1.42 & -13.00 & 5.0e-30 & 0.74 & *** \\
\midrule
\multicolumn{7}{l}{\textit{Agentic RAG vs standard (C2 vs C5)}} \\
  & Accuracy & +0.50 & -5.22 & 4.0e-06 & 0.30 & *** \\
  & Completeness & +0.74 & -6.75 & 9.1e-10 & 0.38 & *** \\
\midrule
\multicolumn{7}{l}{\textit{Enhanced pages (agentic RAG) (C5 vs C6)}} \\
  & Accuracy & +0.34 & -5.08 & 7.7e-06 & 0.29 & *** \\
  & Completeness & +0.40 & -4.89 & 2.0e-05 & 0.28 & *** \\
\midrule
\multicolumn{7}{l}{\textit{Enhanced+ vs Enhanced (agentic RAG) (C6 vs C6+)}} \\
  & Accuracy & +0.06 & -1.47 & 1.0 & 0.08 & n.s. \\
  & Completeness & +0.07 & -1.22 & 1.0 & 0.07 & n.s. \\
\midrule
\multicolumn{7}{l}{\textit{Full pipeline vs baseline (C1 vs C6)}} \\
  & Accuracy & +1.04 & -10.66 & 1.0e-21 & 0.61 & *** \\
  & Completeness & +1.34 & -13.27 & 5.1e-31 & 0.75 & *** \\
\midrule
\multicolumn{7}{l}{\textit{Full pipeline+ vs baseline (C1 vs C6+)}} \\
  & Accuracy & +1.10 & -11.42 & 2.3e-24 & 0.65 & *** \\
  & Completeness & +1.41 & -13.60 & 2.8e-32 & 0.77 & *** \\
\bottomrule
\end{tabular}
\end{table}

\subsection{Analysis by Query Type}

\begin{table}[t]
\centering
\caption{Mean accuracy by condition and query type.}
\label{tab:query_type}
\begin{tabular}{@{}lccccccc@{}}
\toprule
\textbf{Query Type} & \textbf{C1} & \textbf{C2} & \textbf{C3} & \textbf{C4} & \textbf{C5} & \textbf{C6} & \textbf{C6+} \\
\midrule
Factual & 2.74 & 3.14 & 4.57 & 4.06 & 4.13 & 4.56 & \textbf{4.81} \\
Relational & 4.48 & 4.55 & 4.67 & 4.56 & 4.75 & 4.79 & \textbf{4.85} \\
Comparative & 4.77 & 4.94 & \textbf{4.97} & 4.83 & 4.69 & 4.90 & 4.95 \\
\bottomrule
\end{tabular}
\end{table}

Table~\ref{tab:query_type} reveals that the impact of structured data
varies substantially by query type:

\begin{itemize}
    \item \textbf{Factual queries} benefit most from enhanced pages:
          C3 (4.57) vs.\ C1 (2.74), a +66.8\% improvement.
          This is expected, as the enhanced page format embeds entity
          properties and facts in an easily extractable format.
          C6+ achieves the highest factual accuracy (4.81).
    \item \textbf{Relational queries} show consistently high scores
          across conditions (C1: 4.48, C6+: 4.85), suggesting that
          the LLM's pre-trained knowledge handles relational reasoning
          well. Agentic RAG with enhanced pages provides a modest
          additional lift.
    \item \textbf{Comparative queries} show universally high scores
          across all conditions (C1: 4.77, C3: 4.97, C6+: 4.95),
          with enhanced pages providing the most consistent
          near-perfect performance.
\end{itemize}

The most striking finding is the factual query improvement under
enhanced pages: C3 achieves 4.57 accuracy on factual queries compared
to 2.74 for plain HTML, validating the design of our enhanced entity
pages which make entity properties and facts explicitly visible
and extractable.

\subsection{Analysis by Domain}

\begin{table}[t]
\centering
\caption{Mean accuracy by condition and domain.}
\label{tab:domain}
\begin{tabular}{@{}lccccccc@{}}
\toprule
\textbf{Domain} & \textbf{C1} & \textbf{C2} & \textbf{C3} & \textbf{C4} & \textbf{C5} & \textbf{C6} & \textbf{C6+} \\
\midrule
BlackBriar & 4.92 & 4.89 & 4.91 & 4.75 & 4.74 & 4.96 & \textbf{4.99} \\
Express Legal Funding & 3.36 & 4.20 & 4.29 & 4.33 & 4.29 & 4.32 & \textbf{4.86} \\
SalzburgerLand & 2.19 & 2.33 & \textbf{4.92} & 4.06 & 4.25 & 4.82 & 4.66 \\
WordLift Blog & 1.91 & 1.73 & 4.55 & 3.14 & 3.38 & 4.45 & \textbf{4.64} \\
\bottomrule
\end{tabular}
\end{table}

\begin{figure}[H]
\centering
\includegraphics[width=\textwidth]{figures/domain_bars.pdf}
\caption{Mean accuracy by domain and condition. The effect of enhanced
entity pages generalizes across all four verticals, with the largest
gains in editorial (WordLift Blog) and travel (SalzburgerLand) domains.}
\label{fig:domain_bars}
\end{figure}

The domain-level results (Table~\ref{tab:domain} and
Figure~\ref{fig:domain_bars}) demonstrate that
the effect generalizes across all four verticals, while its magnitude
varies with domain characteristics:

\begin{itemize}
    \item \textbf{BlackBriar} (e-commerce) achieves near-perfect
          scores across all conditions (C1: 4.92, C6+: 4.99), likely
          because product entities have well-structured Schema.org
          \texttt{Product}/\texttt{Offer} metadata with clear,
          factual properties that are already well-represented in
          embeddings.
    \item \textbf{SalzburgerLand} (travel) shows the most dramatic
          improvement from enhanced pages: C3 (4.92) vs.\ C1 (2.19),
          indicating that the enhanced page format is particularly
          effective for entities with rich geo-spatial and categorical
          metadata. Agentic RAG also helps: C6 (4.82) vs.\ C4 (4.06).
    \item \textbf{WordLift Blog} (editorial) benefits most from
          enhanced pages: C3 (4.55) and C6+ (4.64) dramatically
          outperform plain HTML conditions (C1: 1.91, C2: 1.73),
          suggesting that editorial content's complex topic
          relationships require explicit structured presentation.
    \item \textbf{Express Legal Funding} (legal) benefits substantially from
          the Enhanced+ format: C6+ (4.86) vs.\ C1 (3.36), showing that
          richer navigational affordances help with complex legal concepts.
\end{itemize}

\subsection{Agentic Metrics}

\begin{table}[t]
\centering
\caption{Agentic-specific metrics across conditions.}
\label{tab:agentic}
\begin{tabular}{@{}lcccc@{}}
\toprule
\textbf{Metric} & \textbf{C4} & \textbf{C5} & \textbf{C6} & \textbf{C6+} \\
\midrule
Links followed & 1.0 & 0.5 & 0.5 & 0.4 \\
Links available & 41.7 & 41.9 & 77.4 & 102.2 \\
\bottomrule
\end{tabular}
\end{table}

\begin{figure}[H]
\centering
\includegraphics[width=0.85\textwidth]{figures/heatmap.pdf}
\caption{Heatmap of mean accuracy scores across domains and conditions.
Darker cells indicate higher accuracy. The pattern shows that enhanced
entity pages (C3, C6) consistently outperform other conditions across
all domains.}
\label{fig:heatmap}
\end{figure}

Table~\ref{tab:agentic} reveals an important finding about
\emph{navigational affordances}: Enhanced+ entity pages (C6+) expose
\textbf{2.4$\times$ more discoverable links} than JSON-LD pages
(102.2 vs.\ 41.9) and \textbf{2.5$\times$ more} than plain HTML
(102.2 vs.\ 41.7). Enhanced entity pages (C6) expose 77.4 links---a
midpoint between C5 and C6+. Interestingly, agents follow \emph{fewer}
links in C6+ (0.4 vs.\ 1.0 in C4), yet achieve higher accuracy.
This suggests that the enhanced page format provides such rich
context in the initial retrieval that the agent needs fewer
additional exploration steps.

The decreasing link-follow rate from C4 (1.0) to C6+ (0.4) suggests
that enhanced pages enable the agent to answer effectively with fewer
actions, indicating more efficient ReAct-style planning when
navigational affordances are clear.

% ===========================================================================
\section{Discussion}
\label{sec:discussion}

\subsection{Two Worlds of AI Search: Parsed vs. Flat Ingestion}

Our findings reveal a critical distinction in the emerging landscape
of AI-powered search---one that has direct implications for the
Generative Engine Optimization (GEO) community.

Today's AI search systems fall into two architectural categories with
respect to structured data:

\paragraph{Dedicated structured data pipelines.}
Traditional search engines such as Google and Bing have evolved
specialized crawl-time parsers that \emph{extract}
\texttt{\textless script type="application/ld+json"\textgreater} blocks
as a separate signal, independent of the page body text. In these
systems, JSON-LD feeds directly into entity understanding, knowledge
panels, and rich results. The structured data is never flattened into
a single text embedding---it is parsed, validated, and indexed in a
dedicated knowledge graph layer.

\paragraph{Flat-text RAG pipelines.}
The fast-growing ecosystem of RAG-based AI assistants, agentic search,
and retrieval-augmented chatbots typically ingests web pages as a
single text chunk. In these systems---which include our Vertex AI
Vector Search 2.0 pipeline as well as most LangChain, LlamaIndex,
and custom RAG deployments---JSON-LD is just more text competing for
a limited embedding budget. Our results show that in this regime,
\textbf{JSON-LD alone provides no measurable benefit} ($\Delta = +0.17$,
$d = 0.18$).

\medskip

This distinction is crucial for practitioners. Our study provides
an empirical snapshot of the \emph{current status quo}: as AI-powered
search diversifies beyond Google and Bing, content optimized only with
JSON-LD will fail to surface in the growing number of flat-text RAG
systems. The enhanced entity page format we propose bridges this gap---it
makes structured knowledge visible and actionable
\emph{regardless of how the search system ingests content}.

\subsection{From SEO to SEO 3.0: The Reasoning Web}

Our findings offer empirical grounding for the emerging concept of
\emph{SEO 3.0}~\cite{volpini2025seo3}, the next evolutionary phase
of search optimization for an AI-driven
web~\cite{gartner2024disruptive}. Search optimization can be
described across three eras:

\begin{itemize}
    \item \textbf{SEO 1.0 --- Document Ranking} (1998--2011): Optimizing
          for keyword matching and link-based authority signals.
          Success measured by ranking position, organic traffic, and
          click-through rates.
    \item \textbf{SEO 2.0 --- Structured Data} (2011--2024): The
          launch of Schema.org in 2011 enabled machines to parse entity
          properties from web pages, powering knowledge panels, rich
          snippets, and improved entity understanding. Success measured
          by structured data adoption, data quality, and rich result
          eligibility.
    \item \textbf{SEO 3.0 --- The Reasoning Web} (2024--present):
          AI search systems do not merely \emph{rank} or \emph{parse}
          content; they synthesize and reason over retrieved information,
          often producing direct answers and taking steps on behalf of
          users. Success must now be measured across three distinct
          dimensions, introduced in the next section.
\end{itemize}

\paragraph{The AI visibility spectrum: Citations, Reasoning, Actions.}
For teams working on AI visibility, our experiment reveals that
content optimization must target three progressively deeper levels of
AI engagement:

\begin{enumerate}
    \item \textbf{Citations} --- \emph{Is your content retrieved and
          attributed?} This is the most basic form of AI visibility:
          appearing as a source in AI-generated answers. Our C1/C2
          conditions evaluate this---whether the retrieval pipeline
          surfaces the right documents. JSON-LD helps with citation
          in systems that parse it (Google, Bing), but not in flat-text
          RAG systems.
    \item \textbf{Reasoning} --- \emph{Can the AI reason correctly
          over your content?} Even when cited, the AI must extract
          and synthesize the right facts. Our accuracy and completeness
          metrics measure this. Enhanced entity pages (C3) improve
          reasoning by +29.5\% over plain HTML---not because the facts
          are different, but because they are \emph{presented} in a way
          the LLM can reliably extract and compose.
    \item \textbf{Actions} --- \emph{Can the AI agent act on your
          content?} In agentic systems, the AI does not just
          retrieve and reason---it follows links, queries APIs, and
          performs multi-step tasks on behalf of the user. Our C4--C6
          conditions evaluate this. Enhanced entity pages with
          dereferenceable URIs and navigational affordances (C6)
          enable agents to traverse knowledge graphs, aggregating
          information across entity boundaries. This is the frontier
          of AI visibility---and it requires content designed for
          \emph{agent traversal}, not just retrieval.
\end{enumerate}

The practical implication for AI visibility teams is clear:
optimizing for \emph{citations alone} (the current focus of most
GEO strategies) is necessary but insufficient. The competitive
advantage lies in optimizing for \emph{reasoning} (through enhanced
content presentation) and \emph{actions} (through navigational
affordances and tool-accessible endpoints).

\subsection{Implications for Web Publishers and GEO}

\begin{enumerate}
    \item \textbf{JSON-LD is necessary but not sufficient}: Schema.org
          markup remains valuable for search engines with dedicated
          parsers (Google, Bing), but it provides \emph{no measurable
          benefit} in RAG-based systems that treat pages as flat text.
          Publishers who rely solely on JSON-LD are optimizing for
          only one class of AI search.
    \item \textbf{Enhanced entity pages are the bridge}: Our format---with
          natural language summaries, navigable links, and
          \texttt{llms.txt}-style agent instructions---achieves
          +29.6\% accuracy in standard RAG and +34.0\% with the full
          Enhanced+ agentic pipeline. It works \emph{in both worlds}: the structured
          JSON-LD is still present for traditional parsers, while the
          human- and agent-readable presentation ensures RAG systems
          can also leverage it.
    \item \textbf{Dereferenceable URIs enable traversal}: Publishing
          entities with URIs that support content negotiation allows
          agents to follow links and aggregate information across
          knowledge graph boundaries---an affordance that flat-text
          pipelines cannot exploit from JSON-LD alone.
\end{enumerate}

\subsection{Implications for RAG System Design}

Our results challenge the dominant ``documents as flat text'' paradigm
and point toward structured-data-aware retrieval:

\begin{enumerate}
    \item \textbf{Presentation matters more than presence}:
          Our experiment shows that the \emph{same underlying knowledge}
          (Schema.org entities) yields dramatically different results
          depending on how it is presented to the retrieval pipeline.
          Raw JSON-LD in a \texttt{\textless script\textgreater} tag:
          no effect. The same information rendered as natural language
          with explicit properties, links, and navigation: +29.5--30.8\%
          accuracy gain.
    \item \textbf{Navigational affordances matter for agents}:
          The gap between C5 and C6 (accuracy: 4.40 vs.\ 4.70)
          shows that even capable agents need explicit navigational
          cues to effectively explore linked data.
    \item \textbf{Enhanced pages enable efficient retrieval}:
          Agents using enhanced pages (C6) make fewer tool calls
          yet achieve higher accuracy, suggesting
          that well-structured content reduces the need for
          multi-hop exploration.
    \item \textbf{Toward structured-data-aware ingestion}: RAG system
          designers should consider architectures that extract and
          separately index structured data blocks---mirroring what
          traditional search engines already do---rather than treating
          all page content as a single text field.
\end{enumerate}

\subsection{Limitations}

Our study has several limitations:
\begin{itemize}
    \item \textbf{Flat-text ingestion architecture}: A critical
          architectural consideration is how our retrieval pipeline
          handles structured data. Vertex AI Vector Search 2.0 ingests
          each document as a single text field (truncated to ${\sim}20$k
          characters for embedding). In our corpus, 82\% of plain HTML
          and 88\% of JSON-LD documents exceed this limit, meaning the
          JSON-LD added to C2 documents is often partially or fully
          truncated before indexing. The JSON-LD \texttt{\textless script\textgreater}
          block starts at a median position of character 18,510---right at
          the truncation boundary.

          \medskip

          This differs fundamentally from how production search engines
          operate. Google's crawler, for example, \emph{extracts} JSON-LD
          from \texttt{\textless script type="application/ld+json"\textgreater} blocks as a
          \emph{separate signal}, independent of the page's body text.
          The structured data is parsed into entity properties and indexed
          in a knowledge graph, not flattened into a single text
          embedding. A retrieval architecture that similarly extracts and
          separately indexes structured data---e.g., using multiple
          embedding fields or a hybrid entity--document store---might
          yield a significant H1 result. This remains an important area
          for future work.

    \item \textbf{Scale}: While 349 queries across 4 domains provides
          strong statistical power, larger-scale experiments would
          further strengthen generalizability claims.
    \item \textbf{LLM judge}: To mitigate correlated biases, we use
          separate models for generation (Gemini 2.5 Flash) and
          evaluation (Gemini 3.0 Flash). While this reduces same-model
          bias, both models share the Gemini family's training
          distribution. Future work should incorporate independent
          human evaluation for additional validation.
    \item \textbf{Single retrieval system}: Our results are specific
          to Vertex AI Vector Search 2.0 with \texttt{gemini-embedding-001}.
          We chose this system intentionally: it represents the
          flat-text ingestion architecture used by most AI search
          systems that operate independently of Google or Bing's
          proprietary index. Systems with structured-data-aware
          ingestion may show different sensitivity to JSON-LD markup.
    \item \textbf{Knowledge graph quality}: Our domains use
          well-maintained knowledge graphs served by WordLift's
          Linked Data Platform. The effectiveness of structured data
          may be lower for noisier or less complete KGs.
\end{itemize}

\subsection{Ethical Considerations and Data Trustworthiness}

A critical aspect of our approach is that the structured data consumed
by AI agents is \textbf{the same data visible to human users}.  The
JSON-LD embedded in each page describes the exact same entity
properties, relationships, and facts that appear in the human-readable
HTML representation.  Similarly, dereferenceable entity URIs serve the
same underlying data through content negotiation---whether rendered as
HTML for humans, JSON-LD for machines, or Turtle for SPARQL queries.

This \textbf{coupling between human and machine representations}
creates stronger faithfulness guarantees than architectures where AI
systems consume entirely separate data feeds.  In a fully decoupled
web---where machines and humans follow two different tracks---there is
no natural accountability mechanism: structured data could drift from
visible content, or be deliberately manipulated to influence AI outputs
without corresponding changes visible to users.  In our approach,
because the structured data \emph{is} the page content (expressed in
machine-readable form), any manipulation would also be visible to human
visitors, creating a natural check on data integrity.

This distinguishes our work from content-optimization approaches such
as GEO~\cite{aggarwal2023geo}, where optimization strategies (adding
statistics, citations, or authoritative language) may create a
divergence between what is optimized \emph{for AI consumption} and what
is genuinely informative \emph{for human readers}.  Our enhanced entity
pages, by contrast, surface the same structured knowledge to both
audiences.

This observation has broader implications for the emerging reasoning
web.  As AI agents increasingly rely on structured data to construct
answers, the trustworthiness of that data becomes paramount.  Systems
that maintain a \textbf{single source of truth}---serving both human
and machine consumers---are inherently more auditable and resistant to
adversarial manipulation than those that decouple the two channels.

\subsection{Future Work}


Our findings motivate several research directions:

\begin{enumerate}
    \item \textbf{Structured-data-aware retrieval}: The null result for
          H1 is specific to our flat-text ingestion pipeline. Future work
          should investigate architectures that extract JSON-LD separately
          and index entity properties as structured metadata---similar to
          how Google treats Schema.org markup as a distinct signal from
          page content. Vertex AI Vector Search 2.0 supports multiple
          data fields with independent embeddings; a dual-field approach
          (body text + structured data) with multi-vector retrieval
          could unlock the latent value of JSON-LD for RAG.
    \item \textbf{Entity-centric chunking}: Rather than truncating
          documents at a fixed character limit, entity-aware chunking
          strategies that preserve structured data blocks could improve
          retrieval for content-rich pages.
    \item \textbf{Cross-system replication}: Replicating our experiment
          with retrieval systems that natively parse structured data
          (e.g., knowledge-graph-augmented retrievers) would help
          disentangle the effect of structured data from the limitations
          of our ingestion pipeline.
    \item \textbf{Production-scale validation}: Deploying enhanced
          entity pages on live websites and measuring their impact on
          AI-powered search engines (SGE, Perplexity) would validate
          ecological validity.
    \item \textbf{Recursive Language Models on Knowledge Graphs}:
          Building on the RLM framework~\cite{zhang2025rlm}, we are
          exploring an approach (RLM-on-KG) that replaces the flat-context
          window with iterative graph exploration~\cite{volpini2025rlmkg}.
          Rather than retrieving a fixed set of documents, the model
          navigates the knowledge graph recursively---fetching thin
          evidence from entity neighbors, deciding which relationships
          to follow, and synthesizing answers with full provenance.
          This extends our current agentic pipeline from tool-augmented
          retrieval to fully recursive, structure-guided reasoning
          over linked data.
\end{enumerate}

\subsection{Practical Recommendations}

Based on our findings, we recommend the following for practitioners:

\begin{enumerate}
    \item \textbf{Go beyond JSON-LD}: While Schema.org markup is
          valuable for search engines that extract it separately, our
          results show it does not improve RAG accuracy when treated
          as flat text. Invest in enhanced entity pages that make
          structured data \emph{human- and agent-readable}.
    \item \textbf{Use dereferenceable URIs}: Ensure that entity URIs
          resolve to content-negotiable endpoints that serve JSON-LD
          when requested programmatically.
    \item \textbf{Adopt the enhanced entity page pattern}: Augment
          existing pages with explicit link navigation, breadcrumbs,
          and \texttt{llms.txt}-style instructions for AI agents.
    \item \textbf{Test with agentic workloads}: As AI-powered search
          becomes prevalent, test content with agentic RAG systems
          rather than relying solely on traditional SEO metrics.
\end{enumerate}

% ===========================================================================
\section{Conclusion}
\label{sec:conclusion}

We have presented a controlled experimental study demonstrating that
enhanced entity pages significantly improve the performance of
Retrieval-Augmented Generation systems. Across 2,439 valid evaluations
spanning four industry domains and seven conditions, we found that:

\begin{enumerate}
    \item Schema.org JSON-LD markup provides only marginal accuracy
          improvements ($\Delta = +0.17$, $d = 0.18$), highlighting
          that structured data must be \emph{presented}, not just embedded.
    \item Our enhanced entity page format, designed for agentic
          discoverability, yields +29.6\% accuracy gains in standard
          RAG ($d = 0.60$) and +29.8\% in the full agentic pipeline
          ($d = 0.61$).
    \item The Enhanced+ variant with richer navigational affordances
          achieves the highest absolute scores (accuracy: 4.85/5,
          completeness: 4.55/5), a +34.0\% improvement over the
          baseline ($d = 0.65$).
    \item Agentic RAG with link traversal significantly improves
          both accuracy (+13.1\%) and completeness (+20.1\%) over
          standard RAG.
    \item These effects generalize across editorial, legal, travel,
          and e-commerce domains.
\end{enumerate}

Our work provides empirical evidence that the Semantic Web's original
vision---machine-readable structured data enabling intelligent
agents---directly translates to measurable improvements in today's
AI systems. As generative AI increasingly mediates information access,
the presence and quality of structured linked data becomes not just
an SEO signal, but a fundamental enabler of accurate, complete,
and well-grounded AI responses.

\paragraph{Reproducibility.} Our dataset, evaluation framework,
enhanced entity page templates, and experiment configuration are
available at \url{https://github.com/wordlift/seo3-reasoning-web}.

\paragraph{Acknowledgements.}
We thank the Google Cloud team for their generous support through Cloud
credits that made these experiments possible.  All experiments---including
embedding generation, vector search indexing, Gemini-based generation
(Gemini 2.5 Flash) and evaluation (Gemini 3 Flash Preview), and the
Vertex AI Vector Search 2.0 infrastructure---were run entirely on Google
Cloud.  We also thank the WordLift engineering team for maintaining the
knowledge graph infrastructure and GraphQL API used in this study.

% ===========================================================================
\bibliographystyle{splncs04}
\bibliography{references}

% ===========================================================================
\appendix

\section{Full Prompts and Templates}
\label{app:prompts}

\subsection{Standard RAG Generation Prompt}

The following prompt is used by Gemini 2.5 Flash to generate answers
from retrieved context in conditions C1--C3 (standard RAG):

\begin{lstlisting}[basicstyle=\small\ttfamily,breaklines=true,frame=single]
You are a helpful assistant answering questions
based on the provided context documents.

IMPORTANT RULES:
1. Answer ONLY based on the information in the
   context documents below.
2. If the context does not contain enough
   information, say "I cannot find sufficient
   information to answer this question."
3. Cite the specific documents you used by
   their ID.
4. Be precise and factual.

CONTEXT DOCUMENTS:
{context}

QUESTION: {query}

Provide a clear, accurate answer with citations
to the source documents.
\end{lstlisting}

\subsection{Agentic RAG System Instruction}

The following system instruction is provided to the Google ADK agent in
conditions C4--C6 (agentic RAG). The agent uses this instruction to plan
its tool-calling strategy:

\begin{lstlisting}[basicstyle=\small\ttfamily,breaklines=true,frame=single]
You are a research assistant with access to a
knowledge graph and document search.
Your goal is to answer the user's question as
accurately and completely as possible.

STRATEGY:
1. First, use search_documents to find relevant
   documents.
2. Examine the results for linked entity URLs
   (especially data.wordlift.io URLs).
3. If the question requires information about
   related entities (provider, offers, etc.),
   use follow_entity_link to fetch their data.
4. If you need to discover entities not directly
   linked, use search_knowledge_graph.
5. Synthesize all gathered information into a
   comprehensive answer.
6. Cite your sources by document ID or entity
   URL.

RULES:
- Only state facts you found in the retrieved
  data. Do not hallucinate.
- Follow at most {max_hops} link hops.
- If you cannot find sufficient information,
  say so explicitly.
\end{lstlisting}

\subsection{LLM Judge Prompts}

Three evaluation prompts are used by the independent judge model
(Gemini 3 Flash Preview). All judge calls request JSON output via
\texttt{response\_mime\_type="application/json"}.

\paragraph{Accuracy (1--5):}
\begin{lstlisting}[basicstyle=\small\ttfamily,breaklines=true,frame=single]
You are an evaluation judge. Given a question,
a ground-truth answer, and a candidate answer
produced by a system, rate the candidate's
factual accuracy.

Question: {question}
Ground Truth Answer: {ground_truth}
Candidate Answer: {candidate}

Rate the accuracy on a scale of 1-5:
  1 = Completely wrong or irrelevant
  2 = Mostly wrong, with minor correct elements
  3 = Partially correct, missing key facts
  4 = Mostly correct, minor inaccuracies
  5 = Fully correct and accurate

Respond in JSON format ONLY:
{"score": <1-5>, "reasoning": "brief explanation"}
\end{lstlisting}

\paragraph{Completeness (1--5):}
\begin{lstlisting}[basicstyle=\small\ttfamily,breaklines=true,frame=single]
You are an evaluation judge. Given a question,
a ground-truth answer containing key facts, and
a candidate answer, rate the completeness of the
candidate.

Question: {question}
Ground Truth (contains key facts the answer
should cover): {ground_truth}
Candidate Answer: {candidate}

Rate completeness on a scale of 1-5:
  1 = Covers none of the key facts
  2 = Covers ~25% of key facts
  3 = Covers ~50% of key facts
  4 = Covers ~75% of key facts
  5 = Covers all key facts

Respond in JSON format ONLY:
{"score": <1-5>, "facts_covered": [...],
 "facts_missing": [...],
 "reasoning": "brief explanation"}
\end{lstlisting}

\paragraph{Grounding (binary):}
\begin{lstlisting}[basicstyle=\small\ttfamily,breaklines=true,frame=single]
You are an evaluation judge. Given a candidate
answer and the source documents it was based on,
determine what fraction of claims in the answer
are traceable to the source documents.

Candidate Answer: {candidate}
Source Documents: {sources}

For each claim in the answer, determine if it
is supported by the source documents.

Respond in JSON format ONLY:
{"grounding_score": <0.0-1.0>,
 "total_claims": <int>,
 "grounded_claims": <int>,
 "ungrounded_claims": ["claim1", "claim2"]}
\end{lstlisting}

\subsection{Enhanced Entity Page Template}

The enhanced entity page template (Jinja2) used for condition C3 and C6
documents. Key features: Schema.org type breadcrumbs, embedded JSON-LD,
visible linked entity navigation with content negotiation instructions,
and \texttt{llms.txt}-style agent instructions.

\begin{lstlisting}[basicstyle=\small\ttfamily,breaklines=true,frame=single,
  language=HTML]
<!DOCTYPE html>
<html lang="en"
  prefix="schema: http://schema.org/">
<head>
  <title>{{ entity_name }}
    -- {{ domain_name }}</title>
  <script type="application/ld+json">
    {{ jsonld_data }}
  </script>
</head>
<body vocab="http://schema.org/"
  typeof="{{ entity_types }}">

  <!-- Type breadcrumbs -->
  <nav class="breadcrumb">
    Thing > CreativeWork > {{ entity_types }}
  </nav>

  <h1 property="name">{{ entity_name }}</h1>
  <div class="entity-type">
    Type: schema:{{ entity_types }}
  </div>

  <section><h2>Description</h2>
    <p property="description">
      {{ entity_description }}
    </p>
  </section>

  <!-- Linked entity navigation -->
  <section class="linked-entities">
    <h2>Related Entities</h2>
    <p>Each URI supports content negotiation
      -- append .json for JSON-LD, .ttl for
      Turtle, or .html for a human-readable
      view.</p>
    <ul>
    
      <li>
        <span class="relation">
          {{ le.relation }}:
        </span>
        <a href="{{ le.url }}">{{ le.url }}</a>
      </li>
    
    </ul>
  </section>

  <!-- Agent instructions (llms.txt style) -->
  <section class="agent-instructions">
    <h2>Agent Instructions</h2>
    <pre>{{ llms_instructions }}</pre>
  </section>

</body></html>
\end{lstlisting}

\subsection{Enhanced+ Entity Page Template (C6+)}
\label{app:enhanced_plus}

The Enhanced+ entity page template extends the base enhanced format
(Appendix~\ref{app:prompts}) with several additional navigational
affordances designed to maximize agentic discoverability:

\begin{itemize}
    \item \textbf{Wikidata-style statements table}: Structured
          property--value pairs displayed in a scannable table format,
          with property names linking to their Schema.org definitions
          and values linking to related entities when applicable.
    \item \textbf{Named linked entities}: Related entity links include
          human-readable names (e.g., ``BlackBriar USA'') instead of
          raw URIs, reducing cognitive load for both agents and humans.
    \item \textbf{Sitelinks section}: Quick-access links to the
          entity's canonical web page and machine-readable serializations
          (JSON-LD, Turtle, RDF/XML).
    \item \textbf{Visual styling}: Card-based layout with distinct
          sections for description, statements, linked entities,
          sitelinks, and agent instructions, improving visual
          hierarchy and scannability.
    \item \textbf{Entity URI display}: Explicit display of the
          entity's dereferenceable URI in the header, making the
          knowledge graph node identity visible.
\end{itemize}

Key structural differences from the base enhanced template:
\begin{lstlisting}[basicstyle=\small\ttfamily,breaklines=true,frame=single,
  language=HTML]
<!-- Statements (Wikidata-style) -->
<table class="statements-table">
  <tr>
    <td class="prop-name">
      <a href="https://schema.org/offers">
        offers</a></td>
    <td><a href="https://data.wordlift.io/
      wl172055/offer/...">$49.99</a></td>
  </tr>
</table>

<!-- Named linked entities -->
<li>
  <span class="relation">brand:</span>
  <a href="/.../blackbriar-usa.html"
     rel="brand">BlackBriar USA</a>
</li>

<!-- Sitelinks -->
<a href="{{ canonical_url }}">Visit Web Page</a>
<a href="{{ entity_url }}.json">JSON-LD</a>
<a href="{{ entity_url }}.ttl">Turtle</a>
\end{lstlisting}

The Enhanced+ template achieves the highest scores in our evaluation
(accuracy: 4.85/5, completeness: 4.55/5), though the improvement over
the base enhanced format (C6) is not statistically significant
($\Delta = +0.06$, $p_\text{adj} = 1.0$), suggesting that the core
enhanced page design captures most of the benefit. The full template
source is available in the repository at
\texttt{templates/enhanced\_entity\_plus.html}.

\end{document}
